\chapter{EXPRESSÕES INDEFINIDAS E INDETERMINADAS}
\label{cap:indet}

\section{Expressão indefinida $\displaystyle \frac{k}{0}$, com  $k \neq 0$}
\label{sec:expindk0}
\quad As expressões algébricas da forma  $\displaystyle \frac{K}{0}$, com $k \neq 0$, não existem, pois considerando um número $\displaystyle x= \frac{k}{0}$, tem-se que $k = x * 0$, sendo por hipótese $k \neq 0$, logo não existe um número para $x$ que satisfaça a igualdade. Isso porque qualquer número multiplicado por zero resultará em zero, portanto a divisão $\displaystyle \frac{k}{0}$ é indefinida ou impossível. Exemplo: $\displaystyle \frac{18}{3} = 6$, isto significa que num total de 18 objetos é possível separa-los em 3 grupos de 6.  Fazendo, $\displaystyle \frac{18}{2} = 9$, isto significa que num total de 18 objetos é possível separa-los em 2 grupos de 9. Mas, $\displaystyle \frac{18}{0}$ não é possível separar os 18 objetos em 0 grupos. Por isso, a divisão por zero não tem significado.


\section{Expressão indeterminada $\displaystyle \frac{0}{0}$}
\label{sec:fra00}
\quad Pode-se pensar no $\displaystyle \frac{0}{0}$ como sendo igual a $1$, tendo em vista que todo número dividido por si mesmo é igual a $1$. Porém ao supor que $\displaystyle \frac{0}{0} = 1$  tem-se que $3 = 3 * 1 = 3 * (\frac{0}{0}) = \frac{(3 * 0)}{0} = \frac{0}{0} = 1$ (uma contradição), e assim por diante pode-se concluir que se trata de uma indeterminação podendo o resultado desta divisão ser qualquer valor real imaginável.\\

Por outro lado, pode-se pensar também que $\displaystyle \frac{0}{0}$ seja igual a $0$, partindo da observação de que $\displaystyle \frac{0}{1} = \displaystyle \frac{0}{2}  = \displaystyle  \frac{0}{3} = etc = 0$. Usando o raciocínio anterior tem-se $\displaystyle \frac{0}{0} = \displaystyle \frac{(2*0)}{0} = \displaystyle 2 *(\frac{0}{0})= 2*0 = 0$, não produzindo uma indeterminação, contudo, a definição $\displaystyle \frac{0}{0} =  0$ também é inaceitável pois produz resultados não naturais.\\

Seja a regra $\displaystyle \frac{(a*b)}{b} = a$, se $b \neq 0$ ficaria diferente para:
$$
\frac{(a*b)}{b} = a \quad \textrm{se} \quad  b \neq 0 \quad \textrm{e} \quad \frac{(a*b)}{b} = 0 \quad \textrm{se} \quad b = 0
$$
Complica-se ao não se levar em conta o valor de $a$, podendo provocar resultados inaceitáveis, do tipo da seguinte descontinuidade de tendência: $\displaystyle y = \frac{x^2-4}{x-2}$ ao aceitar o resultado $\displaystyle \frac{0}{0} = 0$, tem-se:
\begin{displaymath}
 y = \left\{ 
\begin{array}{ll}
\displaystyle \frac{(x+2)(x-2)}{x-2} = x+2 & \textrm{para $x \neq 2$}\\ 
\displaystyle \frac{(x+2)(x-2)}{x-2} = 0 & \textrm{para $x=2$}
\end{array} \right. 
\end{displaymath}

Deste modo, a função ficaria descontìnua em $x=1$. Neste caso é preferível deixar $\displaystyle \frac{0}{0}$ indeterminado em $x=2$, mas com a possibilidade de redefini-la. Isto será visto no próximo capítulo com a introdução de limites.

Analisando as regras aritméticas onde a divisão é o oposto da multiplicação, logo se tem que $\displaystyle \frac{0}{0}$ poderá assumir qualquer valor numérico, pois qualquer número ao ser multiplicado por zero é igual a zero.

\section{Expressão indeterminada $0^0$}

\quad Durante vários séculos, alguns matemáticos como Euler e Cauchy estudaram a polêmica do valor $0^0$. No entanto, a resposta desta polêmica foi revelada somente mais tarde pela contribuição de outros matemáticos.\\

É fácil verificar que  $0^0$ é indeterminado, partindo do resultado que $\displaystyle \frac{a}{a} = a^{1-1} = a^0$, logo $a^0 = 1$ para $a \neq 0$. Quando $a=0$ tem-se que $\displaystyle \frac{0}{0} = 0^0$ que é uma expressão indeterminada por \ref{sec:fra00}.

Também partindo do pressuposto de que $0^0$ é igual a $1$, pois todo número $n^0$ (para $n$ não nulo) ao se forçar $n=0$, levaria a considerar como natural definir $0^0 = 1$. Por outro lado, é razoável também pensar que $0^0$ seja igual a zero, tendo em vista que $0^n = 0$ (para $n$ não nulo), portanto $0^0 = 0$.\\


Existe uma longa discussão entre matemáticos, no entanto, não se pode dar uma resposta universalmente válida para $0^0$, normalmente é mais conveniente definir $0^0 = 1$, porém há situações como no cálculo de limites, onde a prática é considerá-lo como uma forma indeterminada.\\

Existem ainda, outras formas de indeterminações que podem ser facilmente demonstradas, assim:
\section{Expressão indeterminada $\infty -  \infty$}
\label{sec:infinf}
Fazendo 
$$\infty - \infty = \infty (1-1) = \infty * 0 = \infty. \frac{1}{\infty} = \frac{\infty}{\infty} = \frac{\frac{1}{\infty}}{\frac{1}{\infty}} = \frac{0}{0}$$
Como visto em \ref{sec:fra00} uma indeterminação.
\section{Expressão indeterminada $\displaystyle  \frac{\infty}{\infty}$}
Fazendo 
$$
\frac{\infty}{\infty} = \frac{\frac{1}{\infty}}{\frac{1}{\infty}}  = \frac{0}{0} \quad \textrm{(indeterminado)}
$$

\section{Expressão indeterminada $0  . \infty$}
Fazendo 
$$
0 . \infty= \frac{1}{\infty} . \infty  = \frac{\infty}{\infty} = \frac{0}{0} \quad \textrm{(indeterminado)}
$$

\section{Expressão indeterminada $\displaystyle 1^\infty$}
Fazendo 
$$
1^\infty= (\frac{1}{1})^\infty   = \frac{1^\infty}{1^\infty} = 1^{\infty -\infty}
$$

Como se viu em \ref{sec:infinf} $\infty- \infty$ é indeterminado, então $1^\infty$ não pode ser 1.

\section{Expressão indeterminada $\displaystyle \infty^0$}
Fazendo 
$$
\infty^0 = \infty^{1-1} = \frac{\infty}{\infty} = \frac{0}{0}
$$\\



No próximo capítulo será  visto a importância do conceito de limites para o cálculo dessas expressões indeterminadas onde se fará uso de processos intuitivos, artifícios algébricos e geométricos para solucionar essas indeterminações.
