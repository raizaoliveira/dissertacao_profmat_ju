\chapter{GEOGEBRA}
O uso de softwares educativos em ambientes de aprendizagem visa aprimorar a qualidade de ensino, propiciando aos alunos a construção do próprio conhecimento, que segundo Borba e Penteado (2000, p. 10):
\begin{citacao}
O professor pode deixar de ser a única fonte de informações, precisando desempenhar outras funções no sentido de orientar os estudantes na pesquisa de novos conhecimentos e administrar as dificuldades decorrentes do uso de tecnologias e do excesso e dispersão de informações nas redes informáticas.
\end{citacao}
\quad A matemática possui um número privilegiado de softwares educativos, haja vista a necessidade  de se fazer o conteúdo mais atrativo e compreensível, afim de proporcionar melhores resultados. Contudo, escolhemos o software Geogebra como forma de introduzir a informática como ferramenta motivadora no processo ensino aprendizagem, onde neste trabalho, utilizamos este software para análise gráfica no processo de ensino de limite de funções, obtendo através das manipulações uma melhor compreensão do comportamento dessas funções.\\

\quad O Geogebra é um software educacional matemático, livre e dinâmico, criado em 2001, por Markus Hohenwarter, na Universidade Americana Flórida Atlântic University, para ser utilizado em sala de aula, nos estudos das áreas de Álgebra, Geometria e Cálculo.\\


\quad A interface do Geogebra apresenta duas janelas de trabalho uma com informações algébricas e a outra com informações geométricas. Nesse sentido, proporciona uma melhor visualização, pois apresenta simultaneamente as representações algébricas e geométricas de um mesmo objeto, tendo uma interação constante à medida que são fornecidos os comandos no campo de entrada. 
\section{Conhecendo o Programa}
Para baixar a última versão do Geogebra, basta acessar a página do programa :\url{ www.geogebra.org} e fazer o download.

