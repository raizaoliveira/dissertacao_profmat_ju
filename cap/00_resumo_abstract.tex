\begin{resumo}{Indefinidas, Indeterminadas, Limite, GeoGebra}
\label{sec:resumo}
Este trabalho apresenta um estudo sobre expressões indefinidas e indeterminadas no ensino de limites de uma função. Inicialmente é relatada a construção do conceito de limite ao longo da história da matemática, desde os paradoxos de  Zenão até a sua fundamentação teórica com Cauchy e Weirtrass. Em seguida, são apresentadas as expressões indefinidas e indeterminadas e espostas algumas  teorias de limites de uma função por meio de exemplos. Por fim, o software livre e de fácil acesso  \textit{GeoGebra, versão 5.0}, é proposto como ferramenta didático-pedagógica para a manipulação e geração  de gráficos, com intuito  de que o aluno  venha a construir o seu próprio conhecimento sobre limites de uma função.
\end{resumo}
\begin{abstract}{key words}
\label{sec:abstract}
text
\end{abstract}